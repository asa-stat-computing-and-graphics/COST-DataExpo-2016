The GES dataset is a nationally representative sample of
police-reported motor vehicle crashes, released annually. The data is
collected continuously at 60 locations around the United States by the
NHTSA's National Center for Statistics and Analysis. In releasing the
data, the NHTSA intends to provide researchers access to traffic
safety data \citep{NASS}. The GES dataset is a probability sample, and
corresponding to each accident is a complex survey weight. The
incorporation of these weights allows estimation of quantities at the
national level.

We constructed two datasets, covering the time periods 
2005--2009 and 2010--2015 respectively. We chose to consider two disjoint time
periods in order to assess the stability of the cluster
analysis. There is also a significant break between 2009 and 2010 in
the format, structure, and definitions of the NHTSA's GES sample.

\begin{table}[t]
\centering
\caption{Structure of the various data files that form the GES data.}
\label{table_ex1}
\begin{tabular}{@{}llll@{}}
\toprule
Data file   & Multiple records  & Description of            & Unique record  \\ 
name        & per accident      & variables in file         & identifier\\ \midrule
Accident    & No                & Accident characteristics  & Case number  \\
Vehicle     & Yes               & Characteristics of all    & Case number, \\
            &                   & vehicles involved         & vehicle number \\
Person      & Yes               & Characteristics of all    & Case number, \\
            &                   & persons involved          & vehicle number, \\
            &                   &                           & person number\\
Distract    & Yes               & Lists person-level        & same as Person \\
            &                   & distractions              & \\
Visual      & Yes               & Lists person-level        & same as Person  \\
            &                   & visual obstructions       & \\
Impair      & Yes               & Lists person-level        & same as Person  \\
            &                   & impairments               &  \\ \bottomrule
\end{tabular}
\end{table}

All of the variables we employed in our analysis were either indicator
variables for a certain characteristic, or categorical, as shown in
the following list. We selected variables for the two datasets along
the lines of \cite{prato2013bus}.

The binary variables we considered are:
  \begin{itemize}
  \item Accident involved non-motorists;
  \item A school bus was involved;
  \item Located at an intersection;
  \item Single lane road;
  \item Bus driver distracted;
  \item Bus driver impaired or under the influence of alcohol or drugs;
  \item Bus driver speeding;
  \item Bus driver was male;
  \item Pickup truck, Sports Utility Vehicle (SUV), or van involved;
  \item Light / heavy truck involved;
  \item Other driver distracted;
  \item Other driver impaired or under the influence of alcohol or drugs;
  \item Other driver speeding;
  \item Surface conditions adverse (wet, straight, or unlevel road);
  \item Occurred during daylight hours.
  \end{itemize}
  
The categorical variables (more than two categories) we considered are:
  \begin{itemize}
  \item Number of vehicles involved: one, two, three or more;
  \item Speed limit: $<35$ Miles Per Hour (MPH), $35--55$MPH, $>55$MPH;
  \item Critical event that made crash imminent: loss of control,
    vehicle turning, other vehicle in lane, other vehicle encroaching into lane, non-motorist at fault, object or animal, unknown;
  \item Traffic control at crash site: no control, traffic signal,
    traffic sign, other;
  \item Bus movement prior to accident: parking, going straight,
    stopping, decelerating, turning right, turning left, overtaking,
    reversing, negotiating a curve, other.
  \end{itemize}