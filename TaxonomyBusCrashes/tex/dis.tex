
The cluster compositions described above are strikingly stable across
the two datasets. The clusters represent distinct subpopulations of
bus accidents, with clear interpretations and policy implications.

All four clusters represent distinct types of bus accidents. First,
cluster 1 depicts accidents involving non-motorists, such as
pedestrians, cyclists, bicyclists etc. This cluster has a high proportion of accidents at
intersections (59.8\%), which is a reasonable result. Most
pedestrian/cyclist-automobile interactions occur at intersections, due often
to pedestrians/cyclists crossing at cross-walks. To reduce collisions with
non-motorists, then, bus driver training programs can emphasize
particular focus and attention to pedestrians at
intersections. Furthermore, boundaries between automobiles and
sidewalks, like lane lines and curbs, can be accentuated in order to
reduce accidental bus-pedestrian interaction. This cluster also has a high proportion of non-motorists at fault. 

Second, cluster 2 features crashes which happened due to another vehicle encroaching into the bus' lane. Most of
these accidents (57.8\%) occurred in the absence of traffic controls,
and nearly half involved larger vehicles. This may mean that 
these accidents occur on larger roads, and points to accidents that
involve other large vehicles. There are also many school buses in this
cluster, which may be transporting school-children on highways. Also,
13.8\%  of the other drivers were distracted. Taken together, this
cluster indicates the need for increased penalties for distracted
driving, better local awareness by bus drivers, and perhaps mechanical
upgrades to the buses. Proximity sensors that notify drivers of other
cars in their blind spot, or generally adjacent to the vehicle, are
becoming increasingly common in consumer automobiles, and could
prevent lane-changing accidents in this cluster. A less costly,
similar method would be improving visibility of bus turn signals, and
additional mirrors to increase the driver's field of vision.

Third, cluster 3 represents multi-vehicle accidents which happened due to the other vehicle being in the bus' lane. Most of these crashes (54.2\%) occurred when the bus was stopping. This cluster was also the one which involved the highest proportion of school buses (55.1\%). The crashes often involved bigger vehicles (44\%) and in 19.1\% of the cases at least one of the other drivers was caught intoxicated or otherwise impaired. 

Fourth, cluster 4 contains accidents that occur when buses are
turning. This type of accident is unsurprising, given the large
physical profile of buses, which can limit movement. This type of
accident can be discouraged by improving bus driver precision and
driving skill. Also, 19.0\% of these accidents occurred while the bus
driver was distracted, suggesting that driver attentiveness is
important in preventing accidents.


Other than cluster 1, the proportion of accidents in the absence of
traffic controls is roughly 50\%. This initially suggests that bus
accidents might be reduced simply by installing additional traffic
controls. However, this number may simply represent those accidents
that occur on straight roads, and not at a location where more traffic
signals would be useful. A simple categorical variable cannot account
for this nuance, and we hope to clarify these subtypes by
considering conditional combinations of variables---e.g., accidents at
intersections, classified by the presence of traffic controls.

The characteristics of the clusters which change across the two time
periods are highlighted in Table \ref{table:diff}. One of the major
changes for cluster 1---which broadly represents single vehicle
crashes involving non-motorists---is the critical event that caused
the accident. The percentage of crashes due to ``
non-motorist at fault'' increased from 29.0\% to 71.7\%. At the same
time, the proportion of crashes
due to ``other driver under influence'' increased from 4.7\% to
25.7\%. This suggests that non-motorists, possibly pedestrians or cyclists, are
more likely to be intoxicated. The proportion of school buses involved in the accidents included in cluster 1, dropped significantly. More accidents happened at intersections in the later period. This may imply the need for stricter traffic controls at intersections.

We note, too, that the proportion of accidents in cluster 1 occurring in
the absence of traffic controls significantly dropped by half (71.3\% to 36.3\%). We interpret this to indicate that
fewer non-motorists  stray from sidewalks into the street.

For cluster 4---which broadly represents multi-vehicle crashes that
occurred when the bus was turning---the proportion of crashes which
occurred at intersections decreased from 63\% to 53.6\%. At the same
time, the proportion of crashes which occurred in the absence of
traffic controls increased from 41.3\% to 53.3\%, which suggests that
traffic safety officials should examine the presence of traffic
controls at intersections on common bus routes. One encouraging change
that occurred in all clusters is the decrease in the proportion of crashes
due to the bus driver being distracted.

\begin{table}[t]
        \centering
        \caption{Key differences between 2005-2009 and 2010-2015 clusters. Figures are rounded to nearest percentage point.}
        \label{table:diff}
        \resizebox{\textwidth}{!}{
        \begin{tabular}{@{}llll@{}}
                \toprule
                Characteristic               & Cluster         & 2005-2009                                       & 2010-2015 \\  \midrule
                Non-motorist                & 4				 & 0.0\%											& 11.2\% \\
                Bus movement             & 1                  & Parking (15\%)                                  & Parking (2.0\%) \\
                prior to crash                &                     &                          								&  \\  \midrule
                Critical event                & 1                  & Vehicle turning (49.3\%)                          & Vehicle turning (23.5\%) \\ 
                that made the               &                    & Non-motorist                                       & Non-motorist \\
                crash imminent             &                    &  at fault (29\%)                                    & at fault (71.7\%) \\  \midrule
                Speed limit                    & 1                 & 35--55 MPH (54.5\%)                                & 35--55 MPH (74.8\%)   \\
                of the road                     &                   & \textgreater55 MPH (42.7\%)                     &  \textgreater55 MPH (24.3\%) \\ \midrule
                Traffic control                 & 1       & No control (71.3\%)                               & No control (36.3\%)  \\
                devices                           &         & Traffic signal (14.3\%)                           & Traffic signal (44.5\%) \\
                                                       & 4       & No control (41.3\%)                               & No control (53.3\%)  \\   \midrule
                School bus                      & 1       & 40.8\%                                            & 24.6\% \\
                involved?                        &        &                                            &  \\                                              \midrule
                Located at                       & 1       & 42.9\%                                            & 59.8\%  \\
                intersection?                   & 4       & 63.0\%                                            & 53.6\% \\                     \midrule
                Bus driver                        & all     & \multicolumn{2}{l}{Decreased for all clusters by more than 10\%}  \\ 
                distracted?                      &         & \\ \midrule
                Other driver                     & 2       & 25.6\% & 13.8\% \\
                distracted?                       & 3        & 31.0\% & 13.0\%  \\ \midrule
                Other driver                     & 1       & 4.7\% & 25.7\%  \\ 
                under influence?              &         & \\ \midrule
                Daylight?                         & 1       & 89.2\% & 66.0\%  \\\bottomrule
        \end{tabular}
        }
\end{table}
%% limitations: weights in clustering method. COnfounded effect of clustering and changes in population. 

A primary limitation of our analysis is the lack of model-based use of
the sampling weights. Summary statistics and exploratory data analysis
indicate that for many of our variables, the composition remains
consistent between the weighted and unweighted datasets. Although our
exploratory data analysis and visualization of the clusters account
for the sampling weights, our literature search did not indicate
methods for incorporating survey weights in complex clustering methods
such as SOM or neural gas. Consequently, in order to improve the
validity of our assumptions, one avenue for further work is to develop
methods to directly incorporate the weights into both stages of the
clustering approach.

Another limitation is the uncertainty introduced by the GES data
revision in 2009. The differences we observe above between the
2010--2015 and 2005--2009 datasets may reflect a true change in the
taxonomy of bus accidents, but it may also be due to changes in
variable definitions or non-determinism in the clustering
algorithm. However, because the cluster results are so compatible
across the two populations, we are confident that our findings reflect
a stable taxonomy of bus accidents. Furthermore, the stability of the
cluster composition indicates real, latent structure in the data: bus
accidents can be typified by a cluster-based taxonomy, and the
differences between the types are distinct and informative.

Finally, these cluster results are preliminary, and as discussed
above, the simplicity of the clustering features limited the extent to
which we can interpret the clusters for practical use. This indicates
that more detailed and fine-grained analysis of bus taxonomy will
provide yet further understanding of the causes of bus accidents.

In this paper, we constructed a data-driven taxonomy of bus accidents
in the United States using a two-stage clustering method. We
investigated the stability of the cluster composition by assembling
independent taxonomies for two datasets from different time
periods. As anticipated, clearly distinguished accident subtypes are
evident in the data. Furthermore, accompanying these subtypes is a
better picture of the nature and causes of bus accidents. These
results can be used by policy-makers to increase safety regulations
targeted to specific accidents, and allocate funds for improved
traffic controls. These results can also be used to improve training
for bus drivers, inform pedestrians, and influence bus design and
manufacturing. Finally, these results can be used by the researcher as
a base for understanding and taxonomizing bus accident types.

